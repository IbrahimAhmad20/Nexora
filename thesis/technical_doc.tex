% Technical Documentation

\chapter{Technical Documentation}

\section{System Architecture}
The Nexora e-commerce platform follows the three-tier architecture detailed in Chapter 4:
\begin{itemize}
    \item \textbf{Presentation Layer}: Frontend application (Chapter 5)
    \item \textbf{Application Layer}: Backend API (Chapter 6)
    \item \textbf{Data Layer}: MySQL database
\end{itemize}



\section{Development Environment}
\subsection{Required Tools}
\begin{itemize}
    \item Node.js (v14 or higher)
    \item MySQL (v8.0 or higher)
    \item Git
    \item Code editor (VS Code recommended)
    \item Postman (for API testing)
\end{itemize}

\subsection{Development Setup}
\begin{enumerate}
    \item Clone repository:
    \begin{verbatim}
    git clone https://github.com/IbrahimAhmad20/nexora.git
    cd nexora
    \end{verbatim}
    
    \item Install dependencies:
    \begin{verbatim}
    # Backend
    cd backend
    npm install
    
    # Frontend
    cd ../frontend
    npm install
    \end{verbatim}
    
    \item Configure environment:
    \begin{verbatim}
    # Backend (.env)
    NODE_ENV=development
    PORT=3000
    DB_HOST=localhost
    DB_USER=your_username
    DB_PASSWORD=your_password
    DB_NAME=nexora
    JWT_SECRET=your_secret
    
    # Frontend (.env)
    REACT_APP_API_URL=http://localhost:3000
    \end{verbatim}
\end{enumerate}

\section{Code Structure}
\subsection{Frontend Structure}
The frontend codebase (Chapter 5) is organized as follows:
\begin{itemize}
    \item \textbf{src/}
    \begin{itemize}
        \item components/ - Reusable UI components
        \item pages/ - Main application views
        \item services/ - API integration
        \item utils/ - Helper functions
        \item assets/ - Static resources
        \item styles/ - CSS files
    \end{itemize}
\end{itemize}

\subsection{Backend Structure}
The backend codebase (Chapter 6) follows this structure:
\begin{itemize}
    \item \textbf{src/}
    \begin{itemize}
        \item config/ - Configuration files
        \item controllers/ - Request handlers
        \item models/ - Database models
        \item routes/ - API endpoints
        \item services/ - Business logic
        \item utils/ - Helper functions
        \item middleware/ - Custom middleware
    \end{itemize}
\end{itemize}

\section{API Documentation}
\subsection{Authentication}
\begin{verbatim}
POST /api/auth/register
Request:
{
    "username": "string",
    "email": "string",
    "password": "string",
    "role": "customer|vendor|admin"
}

Response:
{
    "id": "string",
    "username": "string",
    "email": "string",
    "role": "string",
    "token": "string"
}
\end{verbatim}

\subsection{Products}
\begin{verbatim}
GET /api/products
Query Parameters:
- page: number
- limit: number
- category: string
- search: string
- sort: string

Response:
{
    "products": [
        {
            "id": "string",
            "name": "string",
            "description": "string",
            "price": "number",
            "category": "string",
            "stock": "number",
            "images": ["string"]
        }
    ],
    "total": "number",
    "page": "number",
    "pages": "number"
}
\end{verbatim}

\section{Database Schema}
The database schema implements the design from Chapter 4:

\begin{verbatim}
-- Users Table
CREATE TABLE users (
    id VARCHAR(36) PRIMARY KEY,
    username VARCHAR(50) NOT NULL,
    email VARCHAR(100) NOT NULL UNIQUE,
    password VARCHAR(255) NOT NULL,
    role ENUM('customer', 'vendor', 'admin') NOT NULL,
    created_at TIMESTAMP DEFAULT CURRENT_TIMESTAMP,
    updated_at TIMESTAMP DEFAULT CURRENT_TIMESTAMP ON UPDATE CURRENT_TIMESTAMP
);

-- Products Table
CREATE TABLE products (
    id VARCHAR(36) PRIMARY KEY,
    name VARCHAR(100) NOT NULL,
    description TEXT,
    price DECIMAL(10,2) NOT NULL,
    category_id VARCHAR(36),
    vendor_id VARCHAR(36),
    stock INT NOT NULL DEFAULT 0,
    created_at TIMESTAMP DEFAULT CURRENT_TIMESTAMP,
    updated_at TIMESTAMP DEFAULT CURRENT_TIMESTAMP ON UPDATE CURRENT_TIMESTAMP,
    FOREIGN KEY (category_id) REFERENCES categories(id),
    FOREIGN KEY (vendor_id) REFERENCES users(id)
);

-- Orders Table
CREATE TABLE orders (
    id VARCHAR(36) PRIMARY KEY,
    user_id VARCHAR(36) NOT NULL,
    total_amount DECIMAL(10,2) NOT NULL,
    status ENUM('pending', 'processing', 'shipped', 'delivered', 'cancelled') NOT NULL,
    created_at TIMESTAMP DEFAULT CURRENT_TIMESTAMP,
    updated_at TIMESTAMP DEFAULT CURRENT_TIMESTAMP ON UPDATE CURRENT_TIMESTAMP,
    FOREIGN KEY (user_id) REFERENCES users(id)
);
\end{verbatim}

\section{Testing Strategy}
\subsection{Unit Testing}
\begin{itemize}
    \item Frontend components (Chapter 5)
    \item Backend services (Chapter 6)
    \item Database models
    \item Utility functions
\end{itemize}

\subsection{Integration Testing}
\begin{itemize}
    \item API endpoints
    \item Database operations
    \item Authentication flow
    \item Payment processing
\end{itemize}

\subsection{End-to-End Testing}
\begin{itemize}
    \item User workflows
    \item Order processing
    \item Admin operations
    \item Error scenarios
\end{itemize}

\section{Deployment Process}
\subsection{Backend Deployment}
\begin{enumerate}
    \item Build application:
    \begin{verbatim}
    cd backend
    npm run build
    \end{verbatim}
    
    \item Configure environment:
    \begin{verbatim}
    NODE_ENV=production
    PORT=3000
    DB_HOST=your-db-host
    DB_USER=your-db-user
    DB_PASSWORD=your-db-password
    DB_NAME=nexora
    JWT_SECRET=your-jwt-secret
    \end{verbatim}
    
    \item Deploy to Render:
    \begin{itemize}
        \item Connect GitHub repository
        \item Configure build settings
        \item Set environment variables
        \item Deploy application
    \end{itemize}
\end{enumerate}

\subsection{Frontend Deployment}
\begin{enumerate}
    \item Build application:
    \begin{verbatim}
    cd frontend
    npm run build
    \end{verbatim}
    
    \item Deploy to Netlify:
    \begin{itemize}
        \item Connect GitHub repository
        \item Configure build settings
        \item Set environment variables
        \item Deploy application
    \end{itemize}
\end{enumerate}

\section{Monitoring and Maintenance}
\subsection{Performance Monitoring}
\begin{itemize}
    \item Response time tracking
    \item Error rate monitoring
    \item Resource usage
    \item Database performance
\end{itemize}

\subsection{Regular Maintenance}
\begin{itemize}
    \item Database backups
    \item Log rotation
    \item Security updates
    \item Performance optimization
\end{itemize}

\section{Troubleshooting Guide}
\subsection{Common Issues}
\begin{itemize}
    \item Database connection errors
    \item Authentication failures
    \item API timeout errors
    \item Build failures
\end{itemize}

\subsection{Resolution Steps}
\begin{enumerate}
    \item Check application logs
    \item Verify environment variables
    \item Test database connectivity
    \item Review error messages
    \item Check system resources
\end{enumerate} 